\documentclass{article}

\usepackage{hyperref}
\usepackage{fancyhdr}
\usepackage[utf8]{inputenc}
\usepackage[TS1,T1]{fontenc}
\usepackage{array, booktabs}
\usepackage{graphicx}
\usepackage[x11names,table]{xcolor}
\usepackage{titling}

\setlength{\droptitle}{-0.75in}

\hypersetup{
colorlinks=false}

\newcommand{\foo}{\color{black}\makebox[0pt]{\textbullet}\hskip-0.5pt\vrule width 1pt\hspace{\labelsep}}

\fancypagestyle{firstpage}{
  \lhead{UC Santa Cruz}
  \rhead{CSE 293: Verilog Project to Silicon - Winter 2025
  }
}

%%%% PROJECT TITLE
\title{Homework 1: UART Arithmetic Logic Unit \\ \large \href{https://github.com/nunibye/cse293/tree/main/hw1}{\underline{github.com/nunibye/cse293/tree/main/hw1}}}

\author{{Eliah Reeves}}

\date{\vspace{-5ex}} %NO DATE

\begin{document}

\maketitle
\thispagestyle{firstpage}
\section{Introduction}

In this assignment I created an Arithmetic Logic Unit (ALU) that can perform a variety of operations on integers. The ALU supports the following operations:
\begin{itemize}
  \item Echo back input.
  \item Addition on a list of 32-bit integers.
  \item Multiplication on a list of 32-bit signed integers.
  \item Division on a pair of 32-bit signed integers.
\end{itemize}
The ALU is designed to be implemented on an iCEBreaker FPGA board and communicates with a host computer via a UART interface. This project is implemented in SystemVerilog.
\section{Procedure}
\subsection{UART Echo}
The first step I took approaching this problem was to implement a UART echo module. For this part I simply connected the RX and TX pine of Alex Forencish's UART modules.
Output waveform of SystemVerilog Simulation:

\begin{figure}[h]
  \centering
  \includegraphics[width=0.7\textwidth]{uart_echo_sim.png}
  \caption{UART echo simulation}
\end{figure}
\newpage
After the success of my initial simulation I ran the same test useing gate level synthesis:

\begin{figure}[h]
  \centering
  \includegraphics[width=0.7\textwidth]{uart_echo_gls.png}
  \caption{UART echo gate level synthesis simulation}
\end{figure}

Finally, the last step was to test the UART echo on the FPGA board. I used a python script and minicom to test.

\begin{figure}[h]
  \centering
  \includegraphics[width=0.6\textwidth]{uart_echo.png}
  \caption{UART echo in minicom}
\end{figure}

Challenges I faced during this part of the assignment were mostly related to getting started. Creating all files and getting something running was difficult. Initially I also set up the UART modules backwards so that the module had data input and output instead of UART input and output. This was a simple fix, but it took some time to figure out. 
\subsection{Sending requests}

The next step in creating a functional ALU was to create a way to test it. To do this I used the serial package to send a request.

\begin{figure}[h]
  \centering
  \includegraphics[width=0.8\textwidth]{bytes_sent.png}
  \caption{Request being sent}
\end{figure}

\newpage

The instruction follows the following format:
\begin{table}[h]
  \centering
  \begin{tabular}{|c|c|c|}
    \hline
    Byte & Field & Description\\
    \hline
    0 & Op code 1 & Specifies operation \\
    \hline
    1 & Reserved & currently unused \\
    \hline
    2 & Length LSB & LSB of command length \\
    \hline
    3 & Length MSB & MSB of command length \\
    \hline
    4-(4-Lenght - 1) & Data & Command payload \\
    \hline
  \end{tabular}
  \caption{Request format}
\end{table}

The ALU supports the following opcodes:
\begin{table}[h]
  \centering
  \begin{tabular}{|c|c|}
    \hline
    Code  & Description\\
    \hline
    0xEC & Echos back all data.  \\
    \hline
    0xAD & Adds a list of signed or unsigned 32-bit integers. \\
    \hline
    0xAF & Multiplies a list of signed 32-bit integers \\
    \hline
    0xA6 & Divides two signed 32-bit integers \\
    \hline
  \end{tabular}
  \caption{Op codes}
\end{table}

This step was not challenging, but it was hard at times to determine if the request was being sent correctly.
\subsection{Implementation}

I Implemented the ALU using a state machine. Theses are the possible states:
\begin{itemize}
  \item GET\_OP\_CODE
This stage checks for a valied op code and saves the operation that much be completed. Goes to GET\_RESERVED.
  \item GET\_RESERVED
This stage doesn't do anything and transitions to GET\_LENGTH\_LSB. Useful for ignoring a byte.
  \item GET\_LENGTH\_LSB
Saves the first byte of the length and goes to GET\_LENGTH\_MSB.
  \item GET\_LENGHT\_MSB.
Saves the rest of the lenghth and goes to GET\_INITIAL\_VALUE or ECHO.
  \item GET\_INITIAL\_VALUE
Saves the first integer into the accumulator. Transitions to GET\_OPERAND.
  \item GET\_OPERAND
Gets next integer and transitions to the operator being performed.
  \item ADD
Adds operand to accumulator. Transitions to GET\_OPERAND or TRANSMIT.
  \item MUL
Multiplies operand and accumulator. Transitions to GET\_OPERAND or TRANSMIT.
  \item DIV
Divides accumulator by operand Transitions to TRANSMIT.
  \item TRANSMIT
Transmits accumulator via UART and transitions to GET\_OP\_CODE.
  \item ECHO
Echos all data received for length number of bytes. Transitions to GET\_OP\_CODE when complete.

\end{itemize}

One challenge I encounter while writing this part of the assignment was that the multiplier never sent ready. I worked around this be resetting the multiplier before using it. This worked, but there is liekly a more elegant way.
\section{Testing}
In order to test my implementation I created a test bench that runs 1000 tests of each operation each test has a random number of inputs with a random value for each input. I also tested the board after it was programmed using minicom and a python script.

\begin{figure}[h!]
  \centering
  \includegraphics[width=0.8\textwidth]{multiply.png}
  \caption{ALU testing multiply}
\end{figure}

\begin{figure}[h!]
  \centering
  \includegraphics[width=0.8\textwidth]{py_test.png}
  \caption{Request sent with Python}
\end{figure}


The test benches were one of the harder parts to write for me. It seemed like the script wasn't waiting correctly, and I had to move edges around a lot to get to send and receive function to work.
\end{document}