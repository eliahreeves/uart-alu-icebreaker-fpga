\documentclass{article}

\usepackage{hyperref}
\usepackage{fancyhdr}
\usepackage[utf8]{inputenc}
\usepackage[TS1,T1]{fontenc}
\usepackage{array, booktabs}
\usepackage{graphicx}
\usepackage[x11names,table]{xcolor}
\usepackage{titling}

\setlength{\droptitle}{-0.75in}

\hypersetup{
    colorlinks=false}
    
\newcommand{\foo}{\color{black}\makebox[0pt]{\textbullet}\hskip-0.5pt\vrule width 1pt\hspace{\labelsep}}

\fancypagestyle{firstpage}{%
  \lhead{UC Santa Cruz}
  \rhead{CSE 293: Verilog Project to Silicon - Winter 2025
  }
}

%%%% PROJECT TITLE
\title{Homework 1: UART Arithmetic Logic Unit}


\author{{Eliah Reeves}}

\date{\vspace{-5ex}} %NO DATE

\begin{document}

\maketitle
\thispagestyle{firstpage}
\section{Introduction}

In this assignment I created an Arithmetic Logic Unit (ALU) that can perform a variety of operations on integers. The ALU supports the following operations:
\begin{itemize}
    \item Echo back input.
    \item Addition on a list of 32-bit integers.
    \item Multiplication on a list of 32-bit signed integers.
    \item Division on a pair of 32-bit signed integers.
\end{itemize}
The ALU is designed to be implemented on an iCEBreaker FPGA board and communicates with a host computer via a UART interface. This project is implemented in SystemVerilog.



\end{document}